\chapter{Introduction}

The Open Source world is extremely varied and permeated by original ideas and innovative technologies. In addition to share free (as in freedom) knowledge it changed the way people write software, improving the quality of the whole software world.

Social coding platforms like \emph{GitHub} allow to easily share code and contribute to other people's project; furthermore, social networks and social media sites are giving a boost to the sharing of ideas and to the exposure of innovative projects.
Nowadays a lot of devices that we use in our everyday life are built on top of an Open Source stack or are themselves open. GNU/Linux is probably the most widespread and known piece of free software, as it runs on almost every device: from huge clusters to mobile phones or even embedded devices. From a recent research by Goldman Sachs the market share of devices running Linux is esteemed to be over 50\%\cite{goldmansachs_microsoft} and more than the 90\% of the top 500 supercomputers run Linux or its variant\cite{top500}.
Thanks to the possibility to modify the source code and expand it with new features, in the last decades a lot of GNU/Linux distributions were born: Gentoo is one of them: born in 2002, it counts now more than 230 voluntary developers spread all over the world. The peculiarity of Gentoo is that it allows the customization of every detail of the OS by making the user compile every piece of software installed on its system: this allows to reach better performance and extremely high levels of personalization. Every Gentoo system can be configured to better fit its use case and exploit the potentialities of the underground hardware.

Gentoo does not offer any binary package; however, Portage, its package manager, handles \emph{ebuilds}: simple scripts that describe how to install a piece of software, from where to fetch the source code to the various steps of the compilation and installation. These \emph{ebuilds} are kept up-to-date by a section of Gentoo developers called \emph{maintainers} and their history is kept on a central repository. Another strong point of Gentoo is that it offers bleeding edge software, as well as stable releases, in order to let the user experiment and play with his system, so more than other distros it needs to be constantly updated and curated.
The number of packages per maintainer may vary but some developers maintain even hundreds of different packages, making their workload very high. It is very difficult to keep an eye on every single package; in fact, following the mailing list of every maintained package is not always feasible. Moreover, all this work is very error prone and does not allow a wide and complete overview of the status of the packages.

Euscan is an application that works in strict contact with Portage and its aim is to detect outdated ebuilds by searching new upstream versions, to offer a complete and insightful overview of the status of the packages, to automate the ebuild maintaining process and to provide a notification system for Gentoo developers.
It was built by developers for developers, with the goal of a better GNU/Linux distribution in mind. All the offered featured have been discussed with the community on IRC or on the mailing lists and some of them were requested directly by Gentoo developers.

I joined the development of Euscan during the Google Summer of Code (GSoC) 2012 with a project initially focused on enhancing its interface. The GSoC is an annual program organized by Google and ideated directly by S. Brin and L. Page, in which students present a proposal of a contribution to a free and open source software among selected organizations and are awarded with stipends if they reach the requested goals. In 2012, 1212 students were accepted in 180 different organizations\cite{gsoc2012}.
After this experience I continued working on other features with the support of the Gentoo community. In February 2013 I became a Gentoo developer and now I am co-maintainer of Euscan and main committer.


The first part of this document will briefly introduce the Gentoo Linux project by presenting its internal structure and its package manager, Portage.
Thereafter the Ebuild Upstream Scanner (Euscan) will be presented in detail with a top down approach, starting from the interface to the upstream version detection algorithm.