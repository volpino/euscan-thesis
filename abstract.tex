\chap{Abstract}

Gentoo is a well known GNU/Linux distribution that focuses primarily on performance and on the possibility for the user to have a very granular customization. Unlike other distributions the user compiles every piece of software according to his favourite settings.

A common requisite for every OS distribution is to offer constantly updated packages, this implies being up-to-date with upstream developers. Major distributions offer tens of thousands of different packages so there is the need of an automated way to track changes and keep offering fresh software.
Euscan (Ebuild Upstream Scanner) is a powerful application for detecting outdated \emph{ebuilds} in the Gentoo package manager by looking for new upstream versions of the packages; this utility is hence important for the quality of the packages offered by Gentoo.

My work aimed at significantly improving the existing software and building brand-new parts in order to be ready-to-use by Gentoo developers. The project focused in particular on adding a task scheduler, a personalized dashboard for Gentoo developers, a new, powerful, effective and flexible algorithm for detecting new upstream versions and other minor useful features.

Euscan was built with the community in mind, so every decision was participated and discussed by Gentoo developers. After this improvement, it became an official Gentoo subproject, it started to be used more and more by the Gentoo team and it will be included soon in the official Gentoo infrastructure.
This project started during the Google Summer of Code 2012 and was carried out with the support of the Gentoo Foundation and the Gentoo developer Corentin Chary in particular.
