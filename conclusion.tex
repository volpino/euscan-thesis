\chap{Conclusions}

Euscan is production-ready and it's used and supported by the Gentoo community. It was built upon ideas collected from other projects and already implemented and widely used in other communities, first of all the Debian one, but introduced new and innovative approaches to the problem, revolutionizing the concept of upstream scanner.

For the future there are many ideas to put in practice and features to improve: for the near future the main task is to migrate Euscan inside the Gentoo infrastructure and let developers login with their Gentoo credentials through LDAP. Then there is the plan of finding an agreement with other developers for standardizing Euscan tags in ebuild metadata. At the moment tag attributes like \emph{versionmangle} are supported by Euscan but are not standardized in metadata DTD.
In the long future the idea is to let Euscan fill version bumps bugs on http://bugs.gentoo.org and automatize all the work done by humans in requesting ebuild updates, however for achieving this major improvement and research should be done.

The whole project is open source and released under the terms of the GNU General Public License version 2, any kind of contribution is encouraged and welcomed.

This composition presents all the core aspects of Euscan, its possible improvement and its limitations. It is also intended to be a good starting point for new developers that would like to contribute to the project, adding new ideas and strength.
I think that Euscan is an useful and important tool for the Gentoo community and I hope it will evolve day by day for better supporting maintainers and users.
